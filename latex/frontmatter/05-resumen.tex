\cleardoublepage
\chapter*{\titulos{RESUMEN}}
\addcontentsline{toc}{chapter}{\normalfont RESUMEN}

El presente trabajo desarrolla un sistema de evaluación adaptativa orientado al contexto universitario, cuyo objetivo es estimar de forma precisa el nivel de conocimiento del estudiante y personalizar la selección de actividades de aprendizaje. El sistema se fundamenta en un modelo híbrido que integra la Teoría de Respuesta al Ítem (IRT) con técnicas de Knowledge Tracing bayesiano, permitiendo un seguimiento continuo del estado de dominio del estudiante.

La arquitectura propuesta se implementa como un motor adaptativo desacoplado, capaz de interactuar con sistemas externos de generación de contenidos educativos. La validación del sistema se realiza mediante simulación computacional con estudiantes virtuales, analizando métricas de precisión, estabilidad y rendimiento. Los resultados obtenidos evidencian la viabilidad técnica del enfoque propuesto y su potencial aplicación en entornos de aprendizaje personalizados.
